% Preamble
\documentclass[11pt]{article}

% Packages
\usepackage{amsmath, amssymb}
\usepackage[numbers]{natbib}


% Document
\begin{document}

\section{LTLf with terminating semantics}

In this interpretation of LTLf, the controller may dictate when a run ends.
This stands in contrast to the former semantics which stated that for a controller
to correctly implement a specification, it has to generate a finite accepting
prefix for each possible input word.

In this new semantics this changes. The controller decides when the word produced
so far verifies the specification and may terminate the sequence at this point.

To implement and model check this, we propose the following.

Each problem is not only described by its LTLf formula but an additional
\textit{extra} atomic proposition (AP), which is obviously not present in
the formula, is given.
We call this special AP terminating signal.

This terminating signal is counted as an output signal and must appear
in the aiger circuit as such.
This terminating signal remains false until a word in the specification
is constructed.
Once it is set to True the current word is terminated, it has therefore
to be accepting and further evaluation of the controller
is undefined behaviour.

That is for every (infinite) sequence $\mathbf{i}$ over the inputs the
controller has to construct sequence $\mathbf{o}$ over the
outputs and the terminating signal $t$ such that there exists a finite
index $k$ with $\forall j < k\colon\ \mathbf{o}_{j} \implies \bar{t}$,
$\mathbf{o}_{k} \implies \bar{t}$ and $\mathbf{i}_{<k}\times\mathbf{o}_{<k}$
verifies the specification.

\subsubsection*{Tool}

The model checking is implemented by the executable called
\textit{ltlfcheck} (system wide installed in the provided docker).
It takes the negation of the specification as a non-deterministic
finite automaton (represented by a state-based büchi automaton in
the hoa format), the controller in the form of an aig-circuit
implementing the specification (as aag file) and finally the name
of the terminating AP.
It then performs symbolic model checking to validate the controller.

Helper functions are implemented via the python script
\textit{ltlf\_fun.py} \\
(/home/ltlfcheck/ltlf\_fun.py), which can be used to
generate controller / negated specification pairs (via some basic,
spot based implementation) or model check these pairs, depending on the arguments.
All actions can be performed batch wise with timeouts.


Generating tests:\\
cd /home/ltlfcheck \&\& mkdir instances\_new \&\& spotrun ltlf\_fun.py generate specifications instances\_new  \\[6pt]

Model check the newly generated pairs\\
cd /home/ltlfcheck \&\& spotrun ltlf\_fun.py modelcheck instances\_new new\_log.txt

\end{document}